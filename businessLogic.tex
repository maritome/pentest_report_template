% Here you can find the list of all test scenarios that are part of the Business Logic Testing category.

% If you want to exclude a test scenario from your report, comment it out, or simply delete it. Removing the scenario from here will not affect its appearance in other places.


\subsection{Business Logic Testing}


% https://owasp.org/www-project-web-security-testing-guide/v42/4-Web_Application_Security_Testing/10-Business_Logic_Testing/01-Test_Business_Logic_Data_Validation.html
\subsubsection{Test Business Logic Data Validation (WSTG-BUSL-01)}
\par
\emph{Test objectives:
\begin{itemize}
    \item Identify data injection points.
    \item Validate that all checks are occurring on the back end and can't be bypassed.
    \item Attempt to break the format of the expected data and analyze how the application is handling it.
\end{itemize}}
% Add the results of the test here. 

% If there is a vulnerability found, generate the vulnerability report table and fill it out with the necessary information.


% https://owasp.org/www-project-web-security-testing-guide/v42/4-Web_Application_Security_Testing/10-Business_Logic_Testing/02-Test_Ability_to_Forge_Requests.html
\subsubsection{Test Ability to Forge Requests (WSTG-BUSL-02)}
\par
\emph{Test objectives:
\begin{itemize}
    \item Review the project documentation looking for guessable, predictable, or hidden functionality of fields.
    \item Insert logically valid data in order to bypass normal business logic workflow.
\end{itemize}}
% Add the results of the test here. 

% If there is a vulnerability found, generate the vulnerability report table and fill it out with the necessary information.


% https://owasp.org/www-project-web-security-testing-guide/v42/4-Web_Application_Security_Testing/10-Business_Logic_Testing/03-Test_Integrity_Checks.html
\subsubsection{Test Integrity Checks (WSTG-BUSL-03)}
\par
\emph{Test objectives:
\begin{itemize}
    \item Review the project documentation for components of the system that move, store, or handle data.
    \item Determine what type of data is logically acceptable by the component and what types the system should guard against.
    \item Determine who should be allowed to modify or read that data in each component.
    \item Attempt to insert, update, or delete data values used by each component that should not be allowed per the business logic workflow.
\end{itemize}}
% Add the results of the test here. 

% If there is a vulnerability found, generate the vulnerability report table and fill it out with the necessary information.


% https://owasp.org/www-project-web-security-testing-guide/v42/4-Web_Application_Security_Testing/10-Business_Logic_Testing/04-Test_for_Process_Timing.html
\subsubsection{Test for Process Timing (WSTG-BUSL-04)}
\par
\emph{Test objectives:
\begin{itemize}
    \item Review the project documentation for system functionality that may be impacted by time.
    \item Develop and execute misuse cases.
\end{itemize}}
% Add the results of the test here. 

% If there is a vulnerability found, generate the vulnerability report table and fill it out with the necessary information.


% https://owasp.org/www-project-web-security-testing-guide/v42/4-Web_Application_Security_Testing/10-Business_Logic_Testing/05-Test_Number_of_Times_a_Function_Can_Be_Used_Limits.html
\subsubsection{Test Number of Times a Function Can Be Used Limits (WSTG-BUSL-05)}
\par
\emph{Test objectives:
\begin{itemize}
    \item Identify functions that must set limits to the times they can be called.
    \item Assess if there is a logical limit set on the functions and if it is properly validated.
\end{itemize}}
% Add the results of the test here. 

% If there is a vulnerability found, generate the vulnerability report table and fill it out with the necessary information.


% https://owasp.org/www-project-web-security-testing-guide/v42/4-Web_Application_Security_Testing/10-Business_Logic_Testing/06-Testing_for_the_Circumvention_of_Work_Flows.html
\subsubsection{Testing for the Circumvention of Work Flows (WSTG-BUSL-06)}
\par
\emph{Test objectives:
\begin{itemize}
    \item Review the project documentation for methods to skip or go through steps in the application process in a different order from the intended business logic flow.
    \item Develop a misuse case and try to circumvent every logic flow identified.
\end{itemize}}
% Add the results of the test here. 

% If there is a vulnerability found, generate the vulnerability report table and fill it out with the necessary information.


% https://owasp.org/www-project-web-security-testing-guide/v42/4-Web_Application_Security_Testing/10-Business_Logic_Testing/07-Test_Defenses_Against_Application_Misuse.html
\subsubsection{Test Defenses Against Application Misuse (WSTG-BUSL-07)}
\par
\emph{Test objectives:
\begin{itemize}
    \item Generate notes from all tests conducted against the system.
    \item Review which tests had a different functionality based on aggressive input.
    \item Understand the defenses in place and verify if they are enough to protect the system against bypassing techniques.
\end{itemize}}
% Add the results of the test here. 

% If there is a vulnerability found, generate the vulnerability report table and fill it out with the necessary information.


% https://owasp.org/www-project-web-security-testing-guide/v42/4-Web_Application_Security_Testing/10-Business_Logic_Testing/08-Test_Upload_of_Unexpected_File_Types.html
\subsubsection{Test Upload of Unexpected File Types (WSTG-BUSL-08)}
\par
\emph{Test objectives:
\begin{itemize}
    \item Review the project documentation for file types that are rejected by the system.
    \item Verify that the unwelcomed file types are rejected and handled safely.
    \item Verify that file batch uploads are secure and do not allow any bypass against the set security measures.
\end{itemize}}
% Add the results of the test here. 

% If there is a vulnerability found, generate the vulnerability report table and fill it out with the necessary information.


% https://owasp.org/www-project-web-security-testing-guide/v42/4-Web_Application_Security_Testing/10-Business_Logic_Testing/09-Test_Upload_of_Malicious_Files.html
\subsubsection{Test Upload of Malicious Files (WSTG-BUSL-09)}
\par
\emph{Test objectives:
\begin{itemize}
    \item Identify the file upload functionality.
    \item Review the project documentation to identify what file types are considered acceptable, and what types would be considered dangerous or malicious.
    \begin{itemize}
        \item If documentation is not available then consider what would be appropriate based on the purpose of the application.
    \end{itemize}
    \item Determine how the uploaded files are processed.
    \item Obtain or create a set of malicious files for testing.
    \item Try to upload the malicious files to the application and determine whether it is accepted and processed.
\end{itemize}}
% Add the results of the test here. 

% If there is a vulnerability found, generate the vulnerability report table and fill it out with the necessary information.