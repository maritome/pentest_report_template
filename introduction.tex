% Here you can write a short intro which contains the test parameters, the methodology and a disclaimer if needed. 

\section{Introduction}


\subsection{Test Parameters}

% The test parameters include the scope, the timeline, the frame conditions, the limitations if any etc. You can add other parameters in the same way as above.

\begin{itemize}
% Enter the name and the url of application under test
  \item[-] \textbf{Application under test}
  \begin{itemize}
      \item[] Name:
      \item[] URL:
  \end{itemize}
% Enter the scope of the performed penetration test
  \item[-] \textbf{Scope}
  \begin{itemize}
      \item[] Penetration Test OWASP Top 10
  \end{itemize}
% Enter the timeline during which the testing took place
  \item[-] \textbf{Testing period}
  \begin{itemize}
      \item[] dd.mm.yyyy - dd.mm.yyyy
  \end{itemize}
% Enter the accounts used and their details
  \item[-] \textbf{User account, roles, and permissions}
  \begin{itemize}
      \item[] 
  \end{itemize}
% Enter the frame conditions
  \item[-] \textbf{Frame Conditions}
  \begin{itemize}
      \item[] 
  \end{itemize}
% Enter the limitations
  % Limitations can be: 
    % Out-of-bounds areas in relation to testing.
    % Broken functionality.
    % Lack of cooperation.
    % Lack of time.
    % Lack of access or credentials.
  \item[-] \textbf{Limitations}
  \begin{itemize}
      \item[] 
  \end{itemize}
\end{itemize}


\subsection{Methodology}

% Enter here a short description of OWASP Web Security Testing Guide and of the risk assessment calculator used.

The test scenarios performed during this penetration test are based on The Open Web Application Security Project (OWASP) Web Security Testing Guide (WSTG)\footnote{\url{https://owasp.org/www-project-web-security-testing-guide/}}, which is a comprehensive guide for testing the security of web applications and web services.
Each test scenario provided by WSTG has an identifier in the format WSTG-category-number, where: 'category' is a 4 character upper case string that identifies the type of test or weakness, and 'number' is a zero-padded numeric value from 01 to 99. For example: WSTG-INFO-02 is the second Information Gathering test.

Based on the international organization National Vulnerability Database (NVD) of the National Institute of Technology (NIST) the assessment of risks is done following the Common Vulnerability Scoring System (CVSS) v3.0\footnote{\url{https://nvd.nist.gov/vuln-metrics/cvss/v3-calculator}}  using Base and Temporal Score Metrics. CVSS is an open framework for communicating the characteristics and severity of software vulnerabilities and its metrics produce a score ranging from 0 to 10 for each vulnerability. This score is also represented as a vector string which is a compressed textual representation of the values used to derive it. The severity ratings defined in the CVSS v3.0\footnote{\url{https://nvd.nist.gov/vuln-metrics/cvss}}  specification are as follows:

\begin{longtable}[c]{|m{4cm}|c|}
\caption{CVSS v3.0 Ratings}
\label{tab:my-table1}\\
\hline
\textbf{Severity} & 
\multicolumn{1}{l|}{\textbf{CVSS Score Range}} \\ 
\hline
\endfirsthead
\endhead
\cellcolor{information}Information & 
0.0 \\ 
\hline
\cellcolor{low}Low & 
0.1 - 3.9  \\ 
\hline
\cellcolor{medium}Medium & 
4.0 - 6.9  \\ 
\hline
\cellcolor{high}High & 
7.0 - 8.9  \\ 
\hline
\cellcolor{critical}Critical & 
9.0 - 10.0 \\ 
\hline
\end{longtable}


% You may wish to provide a disclaimer for your service too. Below is an example provided by OWASP which can be modified based on your agreements and regulations.
\begin{center}
    \large\textbf{Disclaimer}
\end{center}
\textit{This penetration test is a "point in time" assessment and as such the environment could have changed since the test was run. There is no guarantee that all possible security issues have been identified, and new vulnerabilities may have been discovered since the tests were run. Therefore, this report serves as a guiding document and not a warranty which provides a full representation of the risks threatening the systems at hand.}